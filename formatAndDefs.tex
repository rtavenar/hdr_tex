% !TeX root = ./hdr.tex


\usepackage{amsmath,amssymb}             % AMS Math
%\usepackage{oz}
\usepackage[utf8]{inputenc}
\usepackage[T1]{fontenc}
\usepackage{lmodern}
\usepackage{array}
%
\usepackage{enumerate}
\usepackage[shortlabels]{enumitem}

\usepackage{longtable}
\usepackage{etoolbox}
%\usepackage[frenchb]{babel}
%\usepackage[]{subfig}
\usepackage{caption}
\usepackage{subcaption}

\usepackage{nameref}
%\usepackage{breakcites}
\usepackage{cite}
\usepackage{todonotes}
\usepackage[]{amsthm}

\definecolor{ForestGreen}{rgb}{0.0, 0.27, 0.13}
\usepackage{listings}
% Definitions of handy macros can go here
\definecolor{backcolour}{rgb}{0.95,0.95,0.92}
\lstdefinestyle{mystyle}{
    backgroundcolor=\color{backcolour},
    basicstyle=\ttfamily\footnotesize,
    breakatwhitespace=false,
    breaklines=true,
    captionpos=b,
    keepspaces=true,
%    numbers=left,
    numbersep=5pt,
    showspaces=false,
    showstringspaces=false,
    showtabs=false,
    tabsize=2,
    keywordstyle=\color{red!70!black},
    commentstyle=\color{ForestGreen!80!black}
}
\lstset{style=mystyle,language=Python}

\usepackage[ruled]{algorithm2e}


\newcommand{\addref}{\todo[color=red!40]{Référence!}}
\newcommand{\addfig}{\todo[color=red!40]{Figure!} }
\newcommand{\todoi}[1]{\todo[inline]{#1}}
%\newcommand{\to}{\todo[color=red!40]{Add reference.}
% \missingfigure pour faire des figures
\setlength{\marginparwidth}{2cm}

%\usepackage{caption}
% Different font in captions
\newcommand{\captionfonts}{\small}

\makeatletter  % Allow the use of @ in command names
\long\def\@makecaption#1#2{%
  \vskip\abovecaptionskip
  \sbox\@tempboxa{{\captionfonts #1: #2}}%
  \ifdim \wd\@tempboxa >\hsize
    {\captionfonts #1: #2\par}
  \else
    \hbox to\hsize{\hfil\box\@tempboxa\hfil}%
  \fi
  \vskip\belowcaptionskip}
\makeatother   % Cancel the effect of \makeatletter

\usepackage{tikz}
%\usepackage{tkz-graph}
\usetikzlibrary{arrows,automata}

%\usepackage[T1]{fontenc}
\usepackage[left=2.5cm,right=2.5cm,top=2.5cm,bottom=2.5cm,includefoot,includehead,headheight=13.6pt]{geometry}
\renewcommand{\baselinestretch}{1.05}

% Table of contents for each chapter



\newcommand{\refname}{{\sffamily References}}
\usepackage{aecompl}

% Glossary / list of abbreviations

\usepackage[intoc]{nomencl}
\renewcommand{\nomname}{Notations and Acronyms}
\nomlabelwidth=25mm
%\renewcommand{\nomentryend}{\dotfill#1}}
%\renewcommand{\pagedeclaration}[1]{\unskip\dotfill}

%\renewcommand{\nompageref}[2]{\dotfill #1\endgroup}

 \makenomenclature
 \renewcommand{\nomgroup}[1]{%
\ifthenelse{\equal{#1}{N}}{\item[\Large\sffamily\textbf{Notations}]}{%
\ifthenelse{\equal{#1}{X}}{\item[\Large\sffamily\textbf{Acronyms}]}{}}}

\usepackage{makeidx}
 \makeindex

% \usepackage[chapter]{algorithm}
% \usepackage{algorithmic}

\makenomenclature

% My pdf code


\usepackage{ifpdf}

\ifpdf
  \usepackage{graphicx}
  \DeclareGraphicsExtensions{.jpg,.pdf,.png}
  \usepackage[pagebackref,hyperindex=true]{hyperref}
\else
  \usepackage{graphicx}
  \DeclareGraphicsExtensions{.ps,.eps}
  \usepackage[dvipdfm,pagebackref,hyperindex=true]{hyperref}
\fi

\graphicspath{{.}{imgs/}}

%nicer backref links
\renewcommand*{\backref}[1]{}
\renewcommand*{\backrefalt}[4]{%
\ifcase #1 %
(Not cited)%
\or
(Cited on page~#2.)%
\else
(Cited on pages~#2.)%
\fi}
\renewcommand*{\backrefsep}{, }
\renewcommand*{\backreftwosep}{ and~}
\renewcommand*{\backreflastsep}{ and~}

% Links in pdf
\usepackage{color}
\definecolor{linkcol}{rgb}{0,0,0.4}
\definecolor{citecol}{rgb}{0.5,0,0}

% Change this to change the informations included in the pdf file

\hypersetup
{
bookmarksopen=true,
pdftitle=\mytitle,
pdfauthor=\myauthor, %auteur du document
pdfsubject="", %sujet du document
%pdftoolbar=false, %barre d'outils non visible
pdfmenubar=true, %barre de menu visible
pdfhighlight=/O, %effet d'un clic sur un lien hypertexte
colorlinks=true, %couleurs sur les liens hypertextes
pdfpagemode=None, %aucun mode de page
pdfpagelayout=SinglePage, %ouverture en simple page
pdffitwindow=true, %pages ouvertes entierement dans toute la fenetre
linkcolor=linkcol, %couleur des liens hypertextes internes
citecolor=citecol, %couleur des liens pour les citations
urlcolor=linkcol %couleur des liens pour les url
}

\usepackage{pdfpages}
\usepackage{appendix}

% definitions.
% -------------------

\setcounter{secnumdepth}{2}
\setcounter{tocdepth}{2}

% Some useful commands and shortcut for maths:  partial derivative and stuff

\newcommand{\pd}[2]{\frac{\partial #1}{\partial #2}}
\def\abs{\operatorname{abs}}
\def\argmax{\operatornamewithlimits{arg\,max}}
\def\argmin{\operatornamewithlimits{arg\,min}}
\def\diag{\operatorname{Diag}}
\newcommand{\eqRef}[1]{(\ref{#1})}
\newcommand{\lp}[1]{$\ell_{#1}$}

\usepackage{multicol}
\usepackage{rotating}                    % Sideways of figures & tables
%\usepackage{bibunits}
%\usepackage[sectionbib]{chapterbib}          % Cross-reference package (Natural BiB)
%\usepackage{natbib}                  % Put References at the end of each chapter
                                         % Do not put 'sectionbib' option here.
                                         % Sectionbib option in 'natbib' will do.
\usepackage{fancyhdr}                    % Fancy Header and Footer

 %\usepackage{txfonts}                     % Public Times New Roman text & math font

%%% Fancy Header %%%%%%%%%%%%%%%%%%%%%%%%%%%%%%%%%%%%%%%%%%%%%%%%%%%%%%%%%%%%%%%%%%
% Fancy Header Style Options

\pagestyle{fancy}                       % Sets fancy header and footer
\fancyfoot{}                            % Delete current footer settings

% \renewcommand{\chaptermark}[1]{         % Lower Case Chapter marker style
%  \markboth{\chaptername\ \thechapter.\ #1}}{}} %

% \renewcommand{\sectionmark}[1]{         % Lower case Section marker style
%  \markright{\thesection.\ #1}}         %

\fancyhead[LE,RO]{\sffamily\bfseries\thepage}    % Page number (boldface) in left on even
% pages and right on odd pages
\fancyhead[RE]{\sffamily\bfseries\nouppercase{\leftmark}}      % Chapter in the right on even pages
\fancyhead[LO]{\sffamily\bfseries\nouppercase{\rightmark}}     % Section in the left on odd pages

\let\headruleORIG\headrule
\renewcommand{\headrule}{\color{black} \headruleORIG}
\renewcommand{\headrulewidth}{1.0pt}
\usepackage{colortbl}
\arrayrulecolor{black}

\fancypagestyle{plain}{
  \fancyhead{}
  \fancyfoot{}
  \renewcommand{\headrulewidth}{0pt}
}


% \rfoot{\setlength{\unitlength}{1mm}
% \begin{picture}(-0,0)
% \put(-5,-15){\includegraphics[height=2cm]{imgs/animation/image\thepage.png}}
% \end{picture}}


%%% Clear Header %%%%%%%%%%%%%%%%%%%%%%%%%%%%%%%%%%%%%%%%%%%%%%%%%%%%%%%%%%%%%%%%%%
% Clear Header Style on the Last Empty Odd pages
\makeatletter

\def\cleardoublepage{\clearpage\if@twoside \ifodd\c@page\else%
  \hbox{}%
  \thispagestyle{empty}%              % Empty header styles
  \newpage%
  \if@twocolumn\hbox{}\newpage\fi\fi\fi}

\makeatother




%%%%%%%%%%%%%%%%%%%%%%%%%%%%%%%%%%%%%%%%%%%%%%%%%%%%%%%%%%%%%%%%%%%%%%%%%%%%%%%
% Prints your review date and 'Draft Version' (From Josullvn, CS, CMU)
\newcommand{\reviewtimetoday}[2]{\special{!userdict begin
    /bop-hook{gsave 20 710 translate 45 rotate 0.8 setgray
      /Times-Roman findfont 12 scalefont setfont 0 0   moveto (#1) show
      0 -12 moveto (#2) show grestore}def end}}
% You can turn on or off this option.
% \reviewtimetoday{\today}{Draft Version}
%%%%%%%%%%%%%%%%%%%%%%%%%%%%%%%%%%%%%%%%%%%%%%%%%%%%%%%%%%%%%%%%%%%%%%%%%%%%%%%

\newenvironment{maxime}[1]
{
\vspace*{0cm}
\hfill
\begin{minipage}{0.5\textwidth}%
%\rule[0.5ex]{\textwidth}{0.1mm}\\%
\hrulefill $\:$ {\bf #1}\\
%\vspace*{-0.25cm}
\it
}%
{%

\hrulefill
\vspace*{0.5cm}%
\end{minipage}
}


\usepackage{multirow}
%\usepackage{slashbox}
%\usepackage{letterlike}
\newenvironment{bulletList}%
{ \begin{list}%
	{$\bullet$}%
	{\setlength{\labelwidth}{25pt}%
	 \setlength{\leftmargin}{30pt}%
	 \setlength{\itemsep}{\parsep}}}%
{ \end{list} }

\newtheorem{definition}{Definition}
\newtheorem*{hypotheses}{Hypothèses}
\renewcommand{\epsilon}{\varepsilon}
\newtheorem{proposition}{Proposition}[chapter]
\newtheorem{theorem}{Theorem}[chapter]
%\newtheorem{proposition}{Proposition}
% centered page environment

\newenvironment{vcenterpage}
{\newpage\vspace*{\fill}\thispagestyle{empty}\renewcommand{\headrulewidth}{0pt}}
{\vspace*{\fill}}



\newenvironment{rubrique}[2][\linewidth] {
%\styleRub{#2}
\setlength{\lenB}{#1}
\setlength{\lenC}{\linewidth}
\addtolength{\lenC}{-\lenA}
\addtolength{\lenC}{-\lenB}
\addtolength{\lenC}{-\parindent}
\addtolength{\lenC}{-9pt}\vspace{-1mm}
\setlength\itemsep{-1mm}
% %\vspace{- \setlength\itemsep{1em}1.2cm}
%     \ligne{0.1mm}\vspace{-0.5cm}
\indentStd\begin{longtable}[t]{p{\lenB}p{\lenC}}

    }
{\end{longtable}}

\newcommand{\ligne}[1]{\rule[0.5ex]{.96\textwidth}{#1}\\}
\newcommand{\interRubrique}{\bigskip}
\newcommand{\styleRub}[1]{\noindent\textsf{\textbf{\Large #1}}\par}
\newcommand{\indentStd}{\noindent\hspace{\lenA}}



\newcommand{\lieu}[1]{\textsl{#1}}
\newcommand{\activite}[1]{\textbf{#1}}
\newcommand{\comment}[1]{\textsl{#1}}
\newcommand{\papername}[1]{\textsl{#1}}

%%%%%%%%%%%%%%%%%%%%%%%%%%%%%%%%%%%%%%%%%%%%
% Commandes utilisables dans le descriptif %
%
% Modifiables � loisir...
%%%%%%%%%%%%%%%%%%%%%%%%%%%%%%%%%%%%%%%%%%%%





\newcommand{\vectd}{\mathbf{d}}
\newcommand{\vectf}{\mathbf{f}}


% couleur
\newcommand{\bleu}[1]{{\color{blue}{#1}}}
\newcommand{\ver}[1]{{\color{green}{#1}}}
\newcommand{\rouge}[1]{{\color{red}{#1}}}

\newcommand{\pasfini}[1]{\bleu{}\todoi{Pas fini! je te dirai quand
    c'est fait}}


    %\usepackage[nottoc, notlof, notlot]{tocbibind}
\usepackage{minitoc}
\setcounter{minitocdepth}{2}
\mtcindent=10pt

\mtcsetfeature{minitoc}{open}{\vspace{1.5mm}}
\mtcsetfeature{minitoc}{close}{\vspace{1.5mm}}


\let\minitocORIG\minitoc
\renewcommand{\minitoc}{\minitocORIG \vspace{1.5em}}

%nouvelles polices pour minitoc
\renewcommand{\mtcfont}{\sffamily\small}
\renewcommand{\mtcSfont}{\sffamily\small\upshape\bfseries}
\renewcommand{\mtcSSfont}{\sffamily\small}
\renewcommand{\mtcSSSfont}{\sffamily\small}
\renewcommand{\mtifont}{\sffamily\large\bfseries}
\renewcommand{\ptifont}{\sffamily\Huge\bfseries}
% Use \minitoc where to put a table of contents

%\addto\captionsfrench{\def\tablename{{T\scshape{ableau}}}}

%%%%%%%%%%%%%%%%%%%%%%%%%%%%%%%%%%%%%%%%%%%%
% Commandes utilisables dans le descriptif %
%
% Modifiables � loisir...
%%%%%%%%%%%%%%%%%%%%%%%%%%%%%%%%%%%%%%%%%%%%
%%% Local Variables:
%%% mode: latex
%%% TeX-master: "these"
%%% End:
